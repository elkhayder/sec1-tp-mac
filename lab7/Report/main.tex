\documentclass{article}
\usepackage{amsmath}
\usepackage{graphicx} % Required for inserting images
\usepackage{float} % For figures placement
\usepackage{fancyvrb} % for \Verb
\usepackage[a4paper, margin=2.5cm]{geometry}
% Subfigures
\usepackage{caption}
\usepackage{subcaption}
\usepackage{multicol}
\usepackage{geometry}

\usepackage{amsmath}

\usepackage[hidelinks]{hyperref}
\usepackage{tikz-timing}[2017/12/20]


\usepackage{listings}
\usepackage{minted}
\usepackage{xcolor} % to access the named colour LightGray
\definecolor{LightGray}{gray}{0.93}
\setminted{framesep=8mm}
\setminted{baselinestretch=1.2}
\setminted{bgcolor=LightGray}
\setminted{fontsize=\footnotesize}
\setminted{linenos}


\usepackage{setspace} % to change line spacing
\renewcommand{\baselinestretch}{1.5} 

\setlength{\parskip}{\baselineskip}%
\setlength{\parindent}{0pt}%

\begin{document}

\makeatletter
\begin{titlepage}
\begin{center}
    
\includegraphics[width=7cm]{assets/LogoCN_Q.png}
\\
\textbf{\large{Centrale Nantes}}
\\[2cm]

\textbf{\large{MAC : $7^{th}$ Lab's Report \\
Driver For The I/O Expander MCP23S17 (SPI Interface) }}
\\[14pt]
$1^{st}$ year Embedded Systems Engineering
\\[2cm]


\vfill

\textbf{By} \\
EL KHAYDER Zakaria \\
SAOUTI Rayan
\\[1cm]

\textbf{Professor} \\
BRIDAY Mikael
\\[3cm]


February 12, 2023 \\ [12pt]

Session \\
2023-2024 \\[12pt]
\small{Made with \LaTeX}
\end{center}
\end{titlepage}
\makeatother

\pagebreak

\setcounter{page}{1}
\pagenumbering{Roman}

\clearpage
\addcontentsline{toc}{section}{Contents}
\tableofcontents

\clearpage

\addcontentsline{toc}{section}{Figures}
\listoffigures
\addcontentsline{toc}{section}{Listings}
\listoflistings

\clearpage

\setcounter{page}{1}
\pagenumbering{arabic}


\section{SPI Communication}

We have access to \verb|spi.h| that defines these functions:
\begin{itemize}
    \item \textbf{setupSPI}: To initiate and configure the STM32 dedicated SPI chip.
    \item \textbf{beginTransaction}: Start an SPI transaction by lowering the CS pin (CS = 0).
    \item \textbf{endTransaction}: To end the SPI transaction by waiting for all buffer data to be transmitted and then raising back the CS pin (CS = 1).
    \item \textbf{transfer8}: Transfer 1 byte of data using SPI and return the response.
\end{itemize}

Thanks to these already-implemented functionalities, interfacing with the  MCP23S17 will be more about sending the correct data and less about configuring and handling the STM32 signals ourselves.

\subsection{Read/Write}

\begin{figure}[H]
    \centerline{\begin{tikztimingtable}[
        timing/dslope=0.4,
        timing/.style={x=2ex,y=2ex},
        x=5ex,
        timing/rowdist=4ex,
        timing/name/.style={font=\sffamily\scriptsize}]
      SCK    & 50{C} \\
      CS     & H 48L H \\
      MOSI  & 
            1U %
            2D{0} 2D{1} 2D{0} 2D{0} 2D{0} 2D{0} 2D{0} 2D{1} %
            2D{$R_7$} 2D{$R_6$} 2D{$R_5$} 2D{$R_4$} 2D{$R_3$} 2D{$R_2$} 2D{$R_1$} 2D{$R_0$} %
            16U %
            1U \\
            & 1U 16D{OP Code = 0x41} 16D{Register Address} 17U \\
      MISO  & 
            1U 16U 16U %
            2D{$D_7$} 2D{$D_6$} 2D{$D_5$} 2D{$D_4$} 2D{$D_3$} 2D{$D_2$} 2D{$D_1$} 2D{$D_0$} %
            1U \\
            & 1U 16U 16U 16D{Data} 1U \\
    \end{tikztimingtable}}
    \caption{SPI Transaction: Read}
\end{figure}

\begin{figure}[H]
    \centerline{\begin{tikztimingtable}[
        timing/dslope=0.4,
        timing/.style={x=2ex,y=2ex},
        x=5ex,
        timing/rowdist=4ex,
        timing/name/.style={font=\sffamily\scriptsize}]
      SCK    & 50{C} \\
      CS     & H 48L H \\
      MOSI   & 
        1U 2D{0} 2D{1} 2D{0} 2D{0} 2D{0} 2D{0} 2D{0} 2D{0} %
        2D{$R_7$} 2D{$R_6$} 2D{$R_5$} 2D{$R_4$} 2D{$R_3$} 2D{$R_2$} 2D{$R_1$} 2D{$R_0$} %
        2D{$D_7$} 2D{$D_6$} 2D{$D_5$} 2D{$D_4$} 2D{$D_3$} 2D{$D_2$} 2D{$D_1$} 2D{$D_0$} 1U \\
         & 1U 16D{OP Code = 0x40} 16D{Register Address} 16D{Data} 1U \\
      MISO   & 50U \\
    \end{tikztimingtable}}
    \caption{SPI Transaction: Write}
\end{figure}


\subsection{Register Manipulation}

\subsubsection{Registers Enum}
To avoid having to type register addresses by hand, which is both unreadable and error-prone, we chose to declare a \verb|reg| enum that holds the addresses of the 21 different internal registers of the MCP23S17 chip.

\begin{listing}[H]
\begin{minted}{cpp}
enum class reg : uint8_t {
    IODIRA   = 0x00, IODIRB   = 0x01,
    IPOLA    = 0x02, IPOLB    = 0x03,
    GPINTENA = 0x04, GPINTENB = 0x05,
    DEFVALA  = 0x06, DEFVALB  = 0x07,
    INTCONA  = 0x08, INTCONB  = 0x09,
    IOCON1   = 0x0A, IOCON2   = 0x0B,
    GPPUA    = 0x0C, GPPUB    = 0x0D,
    INTFA    = 0x0E, INTFB    = 0x0F,
    INTCAPA  = 0x10, INTCAPB  = 0x11,
    IOA      = 0x12, IOB      = 0x13,
    OLATA    = 0x14, OLATB    = 0x15,
};
\end{minted}
\caption{Registers enum}
\end{listing}

\subsubsection{Write to register}

We can build our custom register write function using the previously mentioned functions from \verb|spi.h|.

We should cast \verb|reg| enum to \verb|uint8_t| since it is the only type accepted by \verb|transfer8|.


\begin{listing}[H]
\begin{minted}{cpp}
void MCP23S17::_writeRegister(reg r, uint8_t val) {
    beginTransaction();

    transfer8(0x40); // 0b01000000
    transfer8((uint8_t)r);
    transfer8(val);

    endTransaction();
}
\end{minted}
\caption{Write to Register}
\end{listing}

\subsubsection{Read from register}

For reading, besides sending both the OP code and the register address, we should send an extra byte so that we can read the data sent by MCP23S17.

\begin{listing}[H]
\begin{minted}{cpp}
uint8_t MCP23S17::_readRegister(reg r) {
    beginTransaction();

    transfer8(0x41); // 0b01000001
    transfer8((uint8_t)r);
    uint8_t val = transfer8(0x00); // Get output

    endTransaction();

    return val;
}
\end{minted}
\caption{Read from register}
\end{listing}


\subsection{Bit Manipulation}

Flipping a single bit (set/reset) can be done in three steps:
\begin{itemize}
    \item Read the existing value
    \item Modify the corresponding bits
    \item Write the value back into the register
\end{itemize}

\begin{listing}[H]
\begin{minted}{cpp}
void MCP23S17::_setBit(reg r, uint8_t idx) {
    uint8_t state = _readRegister(r);
    _writeRegister(r, state | (0x01 << idx));
}
\end{minted}
\caption{Bit Set}
\end{listing}

\begin{listing}[H]
\begin{minted}{cpp}
void MCP23S17::_clearBit(reg r, uint8_t idx) {
    uint8_t state = _readRegister(r);
    _writeRegister(r, state & ~(0x01 << idx));
}
\end{minted}
\caption{Bit Clear}
\end{listing}

\section{MCP23S17 API}

The functions defined in the previous section were not meant to be used by the user directly as they are private \verb|MCP23S17| class methods, rather, they will be used internally by the rest of the methods that we will be declaring, which are meant to be used by the user.

We aim to provide functions similar to the \verb|Arduino| API:
\begin{itemize}
    \item \textbf{pinMode}: Output, Input, Input Pullup
    \item \textbf{digitalRead}
    \item \textbf{digitalWrite}
    \item \textbf{attachInterrupt}: Rising, Falling, Both
\end{itemize}

\subsection{Easier register access}

To simplify our API consumers' lives further, we can declare a \verb|Port| enum (A or B), and hide our previous \verb|reg| enum for internal use only.

\begin{listing}[H]
\begin{minted}{cpp}
enum class Port : uint8_t {
    A = 0x00,
    B = 0x01,
};
\end{minted}
\caption{Port enum}
\end{listing}

We will still use the \verb|reg| enum, although only half of it, to calculate the corresponding Register address for the specific port.

This works wonders with the current register mapping we use (BANK 0) since the registers are only offset with a single address for each port.

\begin{listing}[H]
\begin{minted}{cpp}
MCP23S17::reg MCP23S17::_getRelativeRegister(reg base, Port p) {
    return static_cast<reg>(static_cast<uint8_t>(base) + static_cast<uint8_t>(p));
}
\end{minted}
\caption{Relative Register}
\end{listing}

\subsection{Digital write}

We can write an output on our MCP23S17 by setting the value to the corresponding \verb|GPIO| register

\begin{listing}[H]
\begin{minted}{cpp}
void MCP23S17::digitalWrite(Port p, uint8_t idx, bool value) {
    reg portRegister = _getRelativeRegister(reg::IOA, p);

    if (value) _setBit(portRegister, idx); // 1
    else _clearBit(portRegister, idx); // 0
}
\end{minted}
\caption{digitalWrite}
\end{listing}

\subsection{Read port}

\begin{listing}[H]
\begin{minted}{cpp}
uint8_t MCP23S17::readBits(Port p) {
    return _readRegister(_getRelativeRegister(reg::IOA, p));
}
\end{minted}
\caption{readBits}
\end{listing}

\subsection{Digital read}

\begin{listing}[H]
\begin{minted}{cpp}
bool MCP23S17::digitalRead(Port p, uint8_t idx) {
    return readBits(p) & (0x01 << idx);
}
\end{minted}
\caption{digitalRead}
\end{listing}

\subsection{Pin mode}

According to the MCP23S17 documentation, we should use the corresponding port \verb|IODIR| register. By setting the value of the corresponding bit to \verb|0| the pin would be an \verb|OUTPUT|, otherwise, if it is set to \verb|1|, the pin would be an \verb|INPUT|.

To activate the pullup resistor for a specific pin, we have to set the corresponding bit on the corresponding port \verb|GPPU| register to \verb|1|.

\begin{listing}[H]
\begin{minted}{cpp}
enum class PinMode { Output, Input, Input_Pullup };
\end{minted}
\caption{PinMode Enum}
\end{listing}

\begin{listing}[H]
\begin{minted}{cpp}
void MCP23S17::pinMode(Port p, uint8_t idx, PinMode mode) {
    reg portIoRegister = _getRelativeRegister(reg::IODIRA, p);
    reg portPullupRegister = _getRelativeRegister(reg::GPPUA, p);

    switch (mode)
    {
    case PinMode::Output:
        _clearBit(portIoRegister, idx);
        break;

    case PinMode::Input_Pullup:
        _setBit(portPullupRegister, idx);
        [[fallthrough]]; // Allowing fallthrough to set the pin as input
        
    case PinMode::Input:
        _setBit(portIoRegister, idx);
        break;
    }
}
\end{minted}
\caption{pinMode}
\end{listing}

\subsection{Interrupts}

Interrupts are probably the most complicated part of our MCP23S17 API.

We need to declare a custom \verb|Interrupt struct| where we can save the config (type and handler) for each interrupt.

\begin{listing}[H]
\begin{minted}{cpp}
typedef void (*InterruptCallback)(void);

struct Interrupt {
    bool enabled = false;
    InterruptType type;
    InterruptCallback callback = nullptr;
}
\end{minted}
\caption{Interrupt struct}
\end{listing}

\subsubsection{Interrupt config}

\begin{listing}[H]
\begin{minted}{cpp}
enum class InterruptType { Rising, Falling, Both };
\end{minted}
\caption{Interrupt type}
\end{listing}

We reserve an array of size 16 and type \verb|Interrupt| where we can save our config for the different pins (8 pins per port = 16)

\begin{minted}{cpp}
struct Interrupt _interrupts[16];
\end{minted}

We will also need a helper function that returns an index for each port and pin (hashing if you like)
\begin{listing}[H]
\begin{minted}{cpp}
uint8_t MCP23S17::_getInterruptHandlerIndex(Port p, uint8_t idx) {
    return (uint8_t)p * 8 + idx;
}
\end{minted}
\caption{\_getInterruptHandlerIndex}
\end{listing}

Now with that implemented, we can declare our \verb|attachInterrupt| function that will handle configuring the right register on our MCP23S17 and save the required info in the correct \verb|_interrupts| case.
\begin{listing}[H]
\begin{minted}{cpp}
void MCP23S17::attachInterrupt(Port p, uint8_t idx, InterruptType type, InterruptCallback callback) {
    _clearBit(_getRelativeRegister(reg::INTCONA, p), idx); // Edge interrupt (any)
    _setBit(_getRelativeRegister(reg::GPINTENA, p), idx);  // Enable Interrupt

    _interrupts[_getInterruptHandlerIndex(p, idx)].type = type;
    _interrupts[_getInterruptHandlerIndex(p, idx)].enabled = true;
    _interrupts[_getInterruptHandlerIndex(p, idx)].callback = callback;
}
\end{minted}
\caption{\_getInterruptHandlerIndex}
\end{listing}

As you might notice, we cannot configure the interrupt on a specific edge directly on the MCP23S17, so we will have to handle the edge detection ourselves in our handler. Which makes a nice segue to our next chapter :)


\subsubsection{Interrupt handler}

When there is an interrupt in our MCP23S17, it will trigger an external interrupt in our STM32 uc, the external interrupt pin connected to the MCP23S17 is PA9.

We need to first create an init function where we will set the SPI driver, zero out the \verb|_interrupts| array from whatever garbage existed before, and configure the interrupt on the PA9.

\begin{listing}[H]
\begin{minted}{cpp}
void MCP23S17::Init() {
    setupSPI(); // Setup SPI

    for (uint16_t i = 0; i < sizeof(_interrupts) / sizeof(*_interrupts); i++) {
        _interrupts[i].enabled = false;
        _interrupts[i].callback = nullptr;
    }

    // Configure EXTI (PA9)
    ::pinMode(GPIOA, 9, INPUT_PULLUP);
    ::attachInterrupt(GPIOA, 9, FALLING);
}
\end{minted}
\caption{MCP23S17::Init}
\end{listing}

In our interrupt handler, we will have to get the \verb|INTF| status register state and the \verb|INTCAP| state for both ports. With this information, we can determine if an interrupt occurred on a pin, and what type.
\begin{listing}[H]
\begin{minted}{cpp}
void MCP23S17::onInterrupt(void) {
    uint16_t interruptFlags = _readRegister(MCP23S17::reg::INTFB) << 8 
                            | _readRegister(MCP23S17::reg::INTFA);
    uint16_t state = _readRegister(reg::INTCAPB) << 8 | _readRegister(reg::INTCAPA);

    for (int i = 0; i < 16; i++) {
        uint16_t bit_Msk = 1 << i;
        auto intr = &_interrupts[i];

        // If there is no interrupt flag, or the interrupt is not enabled
        if (!(interruptFlags & bit_Msk) || !intr->enabled)
            continue;

        // Any edge
        if (intr->type == InterruptType::Both)
            intr->callback();
        // Falling Edge
        else if (intr->type == InterruptType::Falling && !(state & bit_Msk))
            intr->callback();
        // Rising Edge
        else if (intr->type == InterruptType::Rising && (state & bit_Msk))
            intr->callback();
    }
}
\end{minted}
\caption{MCP23S17::onInterrupt}
\end{listing}

This function should be called on the actual PA9 interrupt handler \verb|EXTI9_5_IRQHandler|: 


\begin{listing}[H]
\begin{minted}{cpp}
extern "C" void EXTI9_5_IRQHandler(void) {
    ioExt.onInterrupt();
    EXTI->PR |= EXTI_PR_PR9; // acknowledge
}
\end{minted}
\caption{EXTI9\_5\_IRQHandler}
\end{listing}

With this implemented, we have officially implemented all the required and essential functions to facilitate the usage of the MCP23S17 external io expander.

Our result class definition looks like this:
\begin{listing}[H]
\begin{minted}{cpp}
class MCP23S17 {
public:
    typedef void (*InterruptCallback)(void);
    enum class Port : uint8_t { ... };
    enum class PinMode : uint8_t { ... };
    enum class InterruptType { ... };

private:
    enum class reg : uint8_t { ... };

public:
    void Init();

    void pinMode(Port p, uint8_t idx, PinMode mode);
    void digitalWrite(Port p, uint8_t idx, bool value);
    bool digitalRead(Port p, uint8_t idx);
    uint8_t readBits(Port p);

    void onInterrupt(void);

    void attachInterrupt(Port p, uint8_t idx, InterruptType type, InterruptCallback callback);

private:
    void _writeRegister(reg r, uint8_t val);
    uint8_t _readRegister(reg r);
    void _setBit(reg r, uint8_t idx);
    void _clearBit(reg r, uint8_t idx);

    reg _getRelativeRegister(reg base, Port p);
    uint8_t _getInterruptHandlerIndex(Port p, uint8_t idx);
    
private:
    struct Interrupt {
        bool enabled = false;
        InterruptType type;
        InterruptCallback callback = nullptr;
    } _interrupts[16]; // 16 PINS (2 ports, 8 pins each)
};

extern MCP23S17 ioExt;
\end{minted}
\caption{EXTI9\_5\_IRQHandler}
\end{listing}


\section{Sample Application}

Given the simplicity of our application thanks to our beautiful MCP23S17 API, we will not delve into the details of every nook and cranny of it.

\newgeometry{top=1cm, bottom=2.5cm, left=2cm, right=2cm}

\begin{listing}
\begin{minted}{cpp}
enum class Mode { Chaser, BlinkingAll, BlinkingOdd, BlinkingEven };
volatile Mode mode = Mode::Chaser;

void SetModeChaser() { mode = Mode::Chaser; }
void SetModeBlinkingAll() { mode = Mode::BlinkingAll; }
void SetModeBlinkingOdd() { mode = Mode::BlinkingOdd; }
void SetModeBlinkingEven() { mode = Mode::BlinkingEven; }

void ModeChaser() {
    static int i = 0;
    i++;
    if (i > 7)  i = 0;
    else if (i < 0) i = 7;

    for (int j = 0; j < 8; j++) ioExt.digitalWrite(MCP23S17::Port::A, j, i == j);

    for (volatile int x = 0; x < 500000; x++);
}

void ModeBlinking() {
    static bool ledsAreOn = false;
    ledsAreOn = !ledsAreOn;
    uint8_t val;

    switch (mode) {
        case Mode::BlinkingAll:  val = 0xFF; break;
        case Mode::BlinkingEven: val = 0xAA; break;
        case Mode::BlinkingOdd:  val = 0x55; break;
        default: break;
    }

    for (int i = 0; i < 8; i++) {
        if (!ledsAreOn) ioExt.digitalWrite(MCP23S17::Port::A, i, 0);
        else ioExt.digitalWrite(MCP23S17::Port::A, i, val & 1 << i);
    }

    for (volatile int x = 0; x < 1000000; x++);
}

void setup() {
    ioExt.Init();

    for (int i = 0; i < 8; i++) {
        ioExt.pinMode(MCP23S17::Port::A, i, MCP23S17::PinMode::Output);
        ioExt.pinMode(MCP23S17::Port::B, i, MCP23S17::PinMode::Input_Pullup);
    }

    ioExt.attachInterrupt(MCP23S17::Port::B, 4, MCP23S17::InterruptType::Rising, SetModeChaser);
    ioExt.attachInterrupt(MCP23S17::Port::B, 5, MCP23S17::InterruptType::Rising, SetModeBlinkingAll);
    ioExt.attachInterrupt(MCP23S17::Port::B, 6, MCP23S17::InterruptType::Rising, SetModeBlinkingOdd);
    ioExt.attachInterrupt(MCP23S17::Port::B, 7, MCP23S17::InterruptType::Rising, SetModeBlinkingEven);
}

int main() {
    setup();
    while (1) {
        switch (mode) {
            case Mode::Chaser:   ModeChaser();   break;
            default:             ModeBlinking(); break;
        }
    }
}
\end{minted}
\caption{main.c}
\end{listing}

\restoregeometry

\section*{Resources}
\addcontentsline{toc}{section}{Resources}
The code files, and this report's source code, are available on this GitHub repository: \href{https://github.com/elkhayder/sec1-tp-mac}{elkhayder/sec1-tp-mac} 

\end{document}

\end{document}
 